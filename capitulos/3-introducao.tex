\chapter{Introdução}
Atualmente é indiscutível a crescente quantidade de informações científicas que são geradas e armazenadas diariamente. Essas informações encontram-se acessíveis em bancos de dados públicos e privados facilitando o acesso e a utilização em pesquisas científicas. Com a difusão desses dados muitas ciências se fundiram como, por exemplo, a Matemática, a Ciência da Computação e a Biologia Molecular, originando no desenvolvimento de diversas ferramentas computacionais para a análise de dados. Como exemplo de um produto dessa fusão podemos citar o \textit{Big Data} e a Bioinformática.

Devido ao elevado volume de informações relacionadas ao mapeamento genético, ficou inviável a análise manual desses dados. Desta forma, o computador passou a ser uma ferramenta fundamental e abriu espaço para uma nova área de pesquisa, onde os pesquisadores podem realizar seus estudos sem uma máquina sequenciadora ou fazer uso de um laboratório completo \cite{bergeron2003bioinformatics}.

\section{Objetivos}
O objetivo geral desse trabalho consiste na implementação de um algoritmo Hierárquico e o algoritmo \textit{Expectation Maximization} no software ClusterGen. Os seguintes objetivos específicos devem ser realizados para que o objetivo geral seja atingido:
\begin{enumerate}
\item Implementação de um algoritmo 
de clusterização hierárquico;
\item Implementação do algoritmo de  clusterização EM;
\item Comparação dos algoritmos implementados com os métodos já existentes no ClusterGen;
\end{enumerate}