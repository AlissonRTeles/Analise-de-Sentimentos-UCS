\chapter{Introdução}
\label{cap:Introducao}
% contextualizar 
Com a globalização do uso da \textit{internet}, a sociedade aprimorou a sua forma de interagir e desenvolveu novas ferramentas para as relações interpessoais, profissionais e etc. Essas ferramentas são chamadas de redes sociais e são exemplos o \textit{Facebook}\footnote{\url{https://www.facebook.com}}, o \textit{Twitter}\footnote{\url{https://twitter.com}} e o \textit{Instagram}\footnote{\url{https://www.instagram.com}} entre outros no mercado. Essas ferramentas estão presentes na maioria dos \textit{smartphones} da população, e a cada dia aumentam o número de usuários ativos em suas plataformas. Segundo pesquisa realizada pelo \citeonline{ibge2018}, $64,7\%$ das pessoas de 10 anos ou mais utilizaram a \textit{internet} e $94,6\%$ das conexões foram realizadas por \textit{smartphones}. Ainda na mesma pesquisa foi verificado que $94,2\%$ da finalidade dessas conexões foi enviar ou receber mensagens de texto, áudio ou imagens por aplicativos diferentes de \textit{e-mail}. Dentro dessas inúmeras publicações existem muitos conteúdos que são de interesse comum da população já que a mesma está compartilhando seu conhecimento, crítica e opinião aumentando a participação nos mais variados assuntos, desde opinião sobre determinado produto até descontentamento social, há também riscos de agressão verbal entre os membros das comunidades e, o gerenciamento manual acaba sendo impossível devido ao número de dados gerados a cada dia.

% falar sobre o desafio na área 
Para realizar a análise textual dessa imensa quantidade de dados é necessário o uso de recursos computacionais já que a análise manual se torna impraticável  pela demora no tempo de resposta. O número crescente de opiniões também gerou o interesse de grandes empresas em garimpar grupos públicos em interpretar os sentimentos manifestados de seus usuários para verificar aprovação de seus produtos sem necessitar de pesquisas pessoais e até invasivas, diminuindo gastos e acelerando as conclusões para tomadas de decisões. Além disso, há uma necessidade de ferramentas que realizem a análise tendo como base a língua portuguesa já que existem poucas com resultados satisfatórios e poucas pesquisas relacionadas ao idioma.

% apresentar solução
É Nesse contexto que se desenvolve a técnica de \textit{Sentiment Analysis}, uma subárea da TDM (\textit{Text Data Mining})\cite{Hearst:1999:UTD:1034678.1034679}, que se refere ao processo de extrair padrões (conhecimentos de interesse das mais variadas áreas de pesquisa) de documentos textuais não estruturados tendo enfase na detecção das emoções envolvidas nessas produções que podem ter seus sentimentos classificados em: positivos, negativos e neutros. \cite{Li:2010:SAG:2898607.2898826}. 

%bebefícios na area 
As vantagens em fazer pesquisa de opinião por meio de \textit{Sentiment Analysis} são: o baixo custo de análise, rapidez de pesquisa e resposta, autenticidade e não invasão. Sendo os últimos dois itens os maiores diferenciais da técnica pelo fato de as informações usadas como objeto de estudo serem buscadas de comunidades públicas e de a veracidade das informações ser natural já que os usuários não são previamente avisados sobre a pesquisa a ser realizada. Trazendo avanços para a indústria, órgãos públicos, e até mesmo para o indivíduo essa é uma área de pesquisa em crescimento sendo também crescente o interesse em ferramentas ágeis para resultados em tempo real. 

A mesma liberdade de expressão que a \textit{internet} proporcionou deu lugar para que pessoas também utilizem as redes sociais para propagar discursos de ódio contra grupos de minoria. Tornando a \textit{internet} também um lugar hostil com agressões verbais \cite{Chetty2018}.  Além disso, poucos trabalhos abordaram o tema contextualizando o problema na língua portuguesa tornando esse um tema distante e silencioso.

\section{Objetivos}
\subsection{Objetivo Geral}    
    Esse trabalho tem por objetivo desenvolver e avaliar um \textit{software} de detecção de \textit{haters} baseado em Análise de Sentimentos expressos em textos na Língua Portuguesa. 

\subsection{Objetivos Específicos}
    Para atingir o objetivo geral serão seguidos os seguintes objetivos específicos:
\begin{enumerate}
    \item Identificar métodos computacionais para a Análise de Sentimentos e detecção de \textit{haters}.
    \item Implementar um método para detecção de \textit{haters} em dados textuais.
    \item Realizar testes do \textit{software} desenvolvido, comparando os resultados com análises feitas por especialistas humanos.
    \item Avaliar o desempenho do método implementado.
\end{enumerate}
	
\section{Organização do Documento}
O restante do trabalho é organizado da seguinte maneira:

\begin{itemize}

\item \textbf{Capítulo \ref{cap:REFERENCIAL} - Revisão Bibliográfica:} Compreende toda a fundamentação teórica necessária para o desenvolvimento e entendimento do estudo apresentado. É apresentado as etapas principais desenvolvidas em uma análise de sentimento, bem como os principais algoritmos utilizados na classificação dos textos, uma breve explicação dos conceitos de redes sociais, além de estudo de discurso de ódio dentro dessas plataformas. Alguns trabalhos relacionados também são citados no final do capítulo.

\item \textbf{Capítulo \ref{cap:Desenvolvimento} - Desenvolvimento:} São apresentadas as etapas previstas, bem como a metodologia utilizada no desenvolvimento do trabalho tendo o objetivo de criar uma base de léxicos contidos em discursos de ódio, a integração com \textit{software} já desenvolvido que realiza Análise de Sentimentos baseado em emoções e a verificação da efetividade nas detecções de discursos de ódio.

\item \textbf{Capítulo \ref{cap:conclusoes} - Conclusões:} É apresentada síntese do trabalho desenvolvido até então, bem como projeção de atividade futuras em cronograma de atividades detalhado no final do capítulo.
\end{itemize}
