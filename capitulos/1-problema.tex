\chapter{Problema}
\label{cap:PROBLEMA}

A Biologia vem sendo uma ciência observacional e não dedutiva e mesmo que atualmente essa orientação básica não tenha sido alterada, a natureza dos dados mudou radicalmente. Com o progresso das pesquisas biológicas os dados de fato cresceram em quantidade e em precisão.\cite{lesk2005introduction}.


\section{Problema de Pesquisa}
% conceituar usuário sobre o problema de pesquisa
% Para contribuir com o avanço no estudo e manipulação dos dados biológicos, ciências como Matemática, a própria Biologia, Ciência da Computação e outras demais ciências unificaram-se dando origem a Biotecnologia sendo que, para  as funções computacionais, foi criada a Bioinformática. 
%aqui posso tratar dos bancos de dados e suas propriedades e depois falar do problema da demora de consulta de padrões 

Atualmente há uma grande movimentação científica no que diz respeito ao mapeamento genético dos seres vivos e grandes bancos de dados, sejam públicos ou privados, são mantidos por pesquisadores participantes da comunidade acadêmica e também pesquisadores autônomos. O estudo das funções gênicas é uma das principais linhas de pesquisa da Biologia e da Bioinformática pois com o amadurecimento das mesmas será possível compreender os processos moleculares e bioquímicos que melhoram nossa saúde e que causam doenças, para a fabricação de novos remédios e diagnósticos mais confiáveis\cite{Murali2006}. 

Para manipular essas informações e buscar padrões de similaridade genômica, muitas ferramentas computacionais vem sendo desenvolvidas. Elas usam  a combinação de procedimentos matemáticos, estatísticos e de programação. Tudo isso para garantir assertividade já que a sequencia de \textit{DNA} é uma estrutura complexa e crescente.

%posso citar as pesquisas da ucs e depois questionar 

\section{Questão de Pesquisa}
% realizar pergunta a ser respondida com o TCC
Baseado no problema de pesquisa abordado na sessão anterior, foi criada a seguinte questão de pesquisa: É possível acelerar a busca por regiões promotoras e terminadoras de uma mesma base de dados públicos de \textit{DNA} usando técnicas robustas de \textit{clustering}? Os progressos computacionais com a implementação dos novos métodos tornará a ferramenta já implementada pela Universidade de Caxias do Sul mais performática quando exigido um maior número de \textit{clusters}\cite{Basso:2015:UCS}\cite{Fontana:2013:UCS}? 
%falar aqui depois sobre a distribuição dos capítulos