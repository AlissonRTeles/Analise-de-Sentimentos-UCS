\chapter{Conclusões}
\label{cap:conclusoes}
Nesse capítulo são apresentadas as conclusões parciais obtidas na fase de desenvolvimento do Trabalho de Conclusão de Curso I.

\section{Síntese}
\label{sec:sintese}

Esse trabalho apresentou as principais etapas necessárias para a análise de sentimentos. Métodos de coleta de dados, pré-processamento e classificação automática dos dados, abordando os algoritmos e técnicas mais utilizadas nas literaturas recentes sobre o tema

Também foi apresentado breve histórico sobre as redes sociais, o número de usuários e também os tipos de dados que cada uma das ferramentas aceitam. Para o trabalho desenvolvido foi escolhida a rede social \textit{Twitter} por ter textos sintetizados de até $280$ caracteres e porque a mesma tem uma grande quantidade de usuários no mundo todo.

A partir dos trabalhos relacionados foi possível verificar a importância de se desenvolver um dicionário de léxicos na língua portuguesa para casos de discurso de ódio já que é um problema presente na \textit{internet} e que poucos governos punem os \textit{haters} que o praticam.

Pretende-se com este trabalho desenvolver um dicionário de palavras da língua portuguesa utilizadas em discursos de ódio, coletando dados da rede social \textit{Twitter} e realizando o pré-processamento para diminuição de ruídos, e após, classificando com o \textit{software} \textit{Weka} usando os algoritmos \textit{Naive Bayes}, \textit{Bayse Net} e \textit{SVM}, escolhendo o de melhor acurácia conforme classificação manual realizada por especialista humano.

Ao final, será integrada essa nova funcionalidade em um \textit{software} desenvolvido por \citeonline{tccfilipe}. Testes de acurácia serão igualmente realizados para verificar se ferramenta é capaz de identificar discursos de ódio na língua portuguesa e ainda, dizer se é um discurso de ódio contra raça, gênero, deficiência ou religião já que essas são as principais categorias de discurso de ódio segundo estudo realizado por \citeonline{Chetty2018}.

% \section{Cronograma para TCC II}
% \label{sec:cronograma}

% O cronograma do trabalho encontra-se na Tabela \ref{tab-cronograma}, onde as etapas são:

% \begin{enumerate}
% \item Coleta de dados oriundos do \textit{Twitter};
% \item Montagem de base rotulada com especialista humano;
% \item Aplicação de algoritmos de pré-processamento;
% \item Aplicação de algoritmos de \textit{machine learning} no \textit{Weka} (NB, BN, SVM);
% \item Geração de base de léxicos em português;
% \item Realização de testes, coleta e análise dos resultados;
% \item Conclusão do trabalho e identificação das contribuições.

% \end{enumerate}

% \begin{table}[h!]\begin{center}
% 	\caption{Cronograma}\label{tab-cronograma}
% 	\begin{tabular*}{\textwidth}{@{\extracolsep{\fill}} c c c c c c c}
% 		\toprule
% 		& Etapa & mar. & abr. & maio & jun. & jul. &\\
% 		\midrule
% 		&   1   &   X  &      &      &      &      &\\
% 		&   2   &   X  &   X  &      &      &      &\\
% 		&   3   &      &   X  &      &      &      &\\
% 		&   4   &      &   X  &   X  &      &      &\\
% 		&   5   &      &      &   X  &      &      &\\
% 		&   6   &      &      &   X  &      &      &\\
% 		&   7   &      &      &      &  X   &  X   &\\
% 		\bottomrule                             
% 	\end{tabular*}
% \end{center}\end{table}
