% resumo em português
\setlength{\absparsep}{18pt} % ajusta o espaçamento dos parágrafos do resumo
\begin{resumo}

O processo de controle de qualidade de maçãs é realizado geralmente de forma manual por profissionais especializados. De forma a automatizar este processo de inspeção, apresenta-se neste trabalho uma solução para identificação das doenças Olho de Boi e Mancha de Sarna em imagens digitalizadas de maçãs. Estas imagens foram classificadas através do método de Máquina de Vetores de Suporte, o qual tem por objetivo detectar se as maçãs presentes nas imagens são sadias ou possuem as doenças citadas. A implementação foi realizada utilizando-se a biblioteca LibSVM e obteve-se um resultado de 89,26\%.

%Os resultados obtidos serão comparados com uma aplicação de Redes Neurais Artificiais empregada com o mesmo propósito de classificação.

 \textbf{Palavras-chaves}: doença, maçã, Olho de Boi, Mancha de Sarna, classificação, Máquina de Vetores de Suporte, SVM, LibSVM
\end{resumo}

% resumo em inglês
\begin{resumo}[Abstract]
 \begin{otherlanguage*}{english}
   The apple's quality control process is usually performed manually by 
specialized professionals. To automate this inspection process, this paper propose a solution to identify the diseases "Bull's eye rot" and Apple Scab in digitilized apple images. These images were classified using the Support Vector Machine method, which will detect if apples are healthy or presents the mentioned diseases. The implementation was made using LibSVM library and it has achieved up an accuracy of 89,26\%.

   \vspace{\onelineskip}
 
   \noindent 
   \textbf{Keywords}: disease, apple, Bull's eye rot, Apple Scab, classification, Support Vector Machine, SVM, LibSVM
 \end{otherlanguage*}
\end{resumo}