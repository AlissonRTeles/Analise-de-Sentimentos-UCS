% resumo em português
\setlength{\absparsep}{18pt} % ajusta o espaçamento dos parágrafos do resumo
\begin{resumo}

As redes sociais estão cada vez mais inseridas no cotidiano das pessoas. Essas ferramentas trazem melhoria na comunicação, disseminação de opiniões e, várias culturas se unem em meio a postagens sobre os mais variados assuntos. Os brasileiros estão cada vez mais inseridos nos meios tecnológicos e são ativos usuários dessas redes sociais. Junto com a liberdade de expressão vieram comportamentos agressivos de preconceito, os discursos de ódio, tendo como alvos mulheres, deficientes, negros e muçulmanos, espalhando ódio e assédios verbais. Com base nisso, o trabalho pretende abordar sobre a analise de sentimentos aplicada a rede social \textit{Twitter} para detectar discursos de ódio de raça, gênero, religião, e deficiência, especificamente na língua portuguesa já que poucos trabalhos têm abordado sobre a análise de sentimentos e geração de base de léxicos para o idioma. Ao final pretende-se gerar uma base de léxicos na língua portuguesa e realizar testes de acurácia sobre ela.
%Os resultados obtidos serão comparados com uma aplicação de Redes Neurais Artificiais empregada com o mesmo propósito de classificação.

 \textbf{Palavras-chaves}: Análise de Sentimentos, Discurso de Ódio, Twitter, Haters, Weka
\end{resumo}

% resumo em inglês
\begin{resumo}[Abstract]
 \begin{otherlanguage*}{english}
   Social networks are increasingly inserted in people's daily lives. These tools bring improvement in communication, dissemination of opinions and various cultures come together amid postings on the most varied subjects. Brazilians are increasingly inserted in the technological media and are active users of these social networks. Along with freedom of speech came the aggressive behavior of prejudice, hate speech, targeting women, disabled, blacks and Muslims spreading hate and verbal harassment. Based on this, the article intends to approach the sentiment analysis applied to the social network Twitter to detect discourses of hatred of race, gender, religion, and disability, specifically in the Portuguese language, since few studies dealt with the sentiment analysis and generation of base of lexicons for the language. In the end, it is intended to generate a base of lexicons in the Portuguese language and to perform precision tests on it.

   \vspace{\onelineskip}
 
   \noindent 
   \textbf{Keywords}: Sentiment Analysis, Hate Speech, Twitter, Haters, Weka
 \end{otherlanguage*}
\end{resumo}
