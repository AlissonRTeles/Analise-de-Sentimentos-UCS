% resumo em português
\setlength{\absparsep}{18pt} % ajusta o espaçamento dos parágrafos do resumo
\begin{resumo}

As redes sociais estão cada vez mais inseridas no cotidiano das pessoas. Essas ferramentas trazem melhoria na comunicação e disseminação de opiniões, permitindo que várias culturas se unam em meio a postagens sobre os mais variados assuntos. Os brasileiros estão cada vez mais inseridos nos meios tecnológicos e são ativos usuários dessas redes sociais. Contudo, com a liberdade de expressão vieram comportamentos agressivos de preconceito, os discursos de ódio, tendo como alvos mulheres, deficientes, negros e muçulmanos, espalhando ódio e assédios verbais. O controle sobre postagens violentas e de assédio moral é normalmente realizado de forma manual. O uso de ferramentas automatizadas pode auxiliar e facilitar esta tarefa. Com base nisso, o trabalho pretende abordar a Análise de Sentimentos aplicada às redes sociais para detectar discursos de ódio de raça, gênero, religião, e deficiência, especificamente na língua portuguesa. Este tema é relevante já que poucos trabalhos têm abordado a Análise de Sentimentos e geração de base de léxicos para o idioma português. Ao longo deste trabalho pretende-se gerar um dicionário léxico, contendo termos da língua portuguesa, para uso na detecção de \textit{haters}. Tal léxico será usado em um software específico, que uma vez integrado ao \textit{Twitter}, será testado e avaliado.
%Os resultados obtidos serão comparados com uma aplicação de Redes Neurais Artificiais empregada com o mesmo propósito de classificação.

 \textbf{Palavras-chaves}: Análise de Sentimentos, Discurso de Ódio, Twitter, Haters, Weka
\end{resumo}

% resumo em inglês
\begin{resumo}[Abstract]
 \begin{otherlanguage*}{english}
   Social networks are increasingly inserted in people's daily lives. These tools bring improvement in communication and dissemination of opinions, allowing multiple cultures to come together amid postings on the most varied subjects. Brazilians are increasingly inserted in the technological media and are active users of these social networks. However, with freedom of expression came aggressive behavior of prejudice, hate speech, targeting women, disabled, blacks and Muslims, spreading hatred and verbal harassment. Control over violent posting and bullying is usually done manually. The use of automated tools can help and facilitate this task. Based on this, the work intends to approach the sentiment analysis applied to social networks to detect hate speech of race, gender, religion, and disability, specifically in the Portuguese language. This topic is relevant since few studies have dealt with the sentiment analysis and generation of a base of lexicons for the Portuguese language. Throughout this work we intend to generate a lexical dictionary, containing terms of the Portuguese language, for use in the detection of haters. This lexicon will be used in a specific software, which once integrated with Twitter, will be tested and evaluated.

   \vspace{\onelineskip}
 
   \noindent 
   \textbf{Keywords}: Sentiment Analysis, Hate Speech, Twitter, Haters, Weka
 \end{otherlanguage*}
\end{resumo}
